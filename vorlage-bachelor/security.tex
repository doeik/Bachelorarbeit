\chapter{Lxc security}

Linux containers achieve their execution environment security through the following concepts:\\
chroot, namespaces, cgroups, privilege dropping (seccomp policies, app armor)\\

\section{Cgroups}

Cgroups is the feature of choice when it comes to limiting machine resources and as such preventing dos attacks.
Also it is not pre-configured in any way, so this needs some accounting:\\
First thing to do is to to limit the memory to prevent processes in the container from eating up the whole machine memory.
When the memory consumtion reaches the specified threshold the cgroups memory resource controller tries to reclaim memory first,
but if that fails it will kill the most memory consuming process in the container which should result in a return code of 9.
I have tested this with a minimal c program which consumes as much memory as possible.\\
If the system has got swap memory this must also be limited. Otherwise it would just begin to swap out after the memory
threshold has been reached. Since the kernel does not distinguish between swapped or unswapped memory the limit computes as follows:
memory limit + swap limit.\\
Currently my daemon does not determine whether the swap is activated or not, this needs to be adjusted manually in the cgroups
section of the config template by uncommenting the line stating lxc.memory.msw.limit\_in\_bytes.\\
For IO operations cgroups do mostly read/write rate throttling, so there is no limit restricting the amount of disk space programs
can use up. However this is not a huge problem since the lxc\_daemon always sets a timeout on container executions after which the
rootfs is destroyed and therefore all disk space freed. It is unlikely that this will have an impact on the system's health although
running several containers at once will have a significant impact on how fast the execution will terminate especially if all these
need to do IO operations.\\
One of the severest dos attacks is the fork bomb because it causes the system to freeze completely. Fortunately cgroups provides
the kmem extension with which it is possible to limit the kernel memory to a small amount. If set properly this will cause the fork()
system call to fail way before freezing the system.

Currently the memory.kmem feature is disabled in the default archlinux kernel due to the current development state.\cite{kmembug}

\section{Capability dropping}

Since the goal is to allow c-programs by students that are only granted some standart ansi c libraries, capability dropping can be
done somewhat relentlessly. Nevertheless some capabilities are more significant than others and need to emphasized:\\
Setpcap is an important capability to drop. Owning this capability would allow a possibly harmful program to re-grant itself any
capability.\cite{kernelcaps}\\
The mount capability would allow to forego apparmor restrictions since these are bound to specific paths.

\section{Seccomp policies}

Seccomp policies are a rather low level feature, hence it is not safe tampering with them. They depend on system specific information such
as the architecture (x64 or x86). Essentially they are currently used by lxc to blacklist some unsafe system calls.