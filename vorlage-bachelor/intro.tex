\chapter{Introduction}

Programs are capable of harming the operating system in various ways. They are able to constrain the system in a way that
it may become inoperable by destroying crucial files, or simply deny its resources to other processes.
The latter is called \textit{Denial of Service} (DoS).
They might furthermore try to gather sensitive information or use its resources for their own purposes.\\
There are several solutions trying to increase the security of the operating system when executing untrusted software,
three of them being mentioned here:\\
Container virtualization with \textit{Linux containers}\cite{lxc}, a virtual machine like \textit{qemu}\cite{qemu} and \textit{ptrace}\cite{ptrace}.\\
The first two both try to achieve security by isolating the potentially malicious program from the host system.
\textit{Ptrace} instead is capable of inspecting and intercepting system calls. It is able to modify memory and registers of a process that it is attached to.\\
This thesis is focusing on analyzing container virtualization because it ideally provides an execution environment for
potentially malicious code which is entirely isolated from the rest of the system while being comparatively
lightweight. Unlike virtual machines which come with emulated hardware and a completely separate operating
system, container virtualization works with the host's kernel and a set of security features provided by the
\textit{Linux} kernel (see chapter \ref{features} for a full list of these).
This same concept which provides a huge advantage over fully virtualized machines has also a major downside
because it restricts -- unlike  \textit{VMs} -- an implementation to a single kernel (which has to provide sufficient
security features of course). Nevertheless, it takes just very few seconds to build and start a container.
The \textit{Linux Container} project (\textit{lxc}) is one of the existing implementations for container virtualization,
however it takes a notable amount of time to understand and configure it. With this in mind, the intention was to create a program
which manages creation, execution and destruction automatically, is safely configured and easy to set up.\\
Another project with almost similar goals is \textit{docker}\cite{docker} which also focuses on usability but
additionally has grown into a multi platform application.
This project shall be used for an automated system that runs tests on C project files from students in the ``info 2'' module
at \textit{Heinrich Heine University} in Düsseldorf.