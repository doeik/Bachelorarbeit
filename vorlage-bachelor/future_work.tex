\chapter{Future work}

\section{Unprivileged containers}

\textit{Lxc} can also be run by a non root user which would make the \texttt{lxc\_daemon} obsolete.
However the process is both internally and externally quite different from privileged execution.
Container creation cannot be done with the standart templates because these use tools like
\textit{debootstrap} which need to do some privileged operations. A solution to that is the \texttt{download} template
which pulls any supported rootfs that is updated daily. Furthermore the config would need to
be adjusted to go with unprivileged containers, plus the user namespace would become essential
although it is still not enabled in some distributions (\textit{Arch Linux} being one of them).
Last but not least the owner of the rootfs needs to be changed to match the user who is executing
the container which is still a privileged operation.

\section{Daemon actions}

At the moment the daemon features only one single action. It would be convenient for practical use
to be able to export files that are created during the process to the unprivileged peer.
Also an action which does not run a program directly but rather passes it as a parameter to a different
command (e.g. gcc) would come in handy.