\chapter{Daemon structure}

\section{Purpose}

Since lxc requires the processing user to have root privileges, precautions need to be taken to ensure
the safety of the system. Mainly we do not want to have root privileges for more actions other than those who really
need these. As a consequence programs are passed to the lxc\_daemon who then performs the container creation
and execution and in the end returns the results back to an unprivileged program (e.g. the mail server).

\section{Communication}

Any program who wants to communicate with the lxc\_daemon is able to do this through a unix domain socket located
in `/run'. Unix domain sockets are mostly handled like IP-sockets but instead of using an IP address UDS take file paths.
Also access to this socket can be limited easily through altering the file modes.
Practically speaking this allows us to only permit programs with a certain group or user id to access the socket which then again
implies that an admin has explicitly allowed this user to be able to communicate with the daemon.\\
Currently the name of the usergroup is ``info2\_containers''.\\
When not running the lxc\_daemon as a system service the option ``-g'' needs to be specified. This sets the group id
accordingly and thus preventing the file modes from being set incorrectly.

\section{The protocol}

The daemon takes a json object containing a python dictionary which needs to specify at least a valid action to
be performed as well as some arguments which are needed for that particular action.\\
Currently the daemon only performs the ``run\_prog'' action which requires a base64 encoded program to be run
inside the container under the keyword ``b64\_data''. Note that the base64 byte string needs to be converted to
UTF8 first as json complains when a byte string is passed.\\
In addition to that this action requires a list of parameters under the keyword ``params'' which is obligatory
because the first argument is the name of the program.\\
There are a few other non-essential options:\\
The ``keep-alive'' keyword (default: False) contains a boolean indicating whether the connection is to be closed
directly after this action or to be held open awaiting further instructions.\\
The ``timeout'' keyword (default: TIMEOUT) specifies after how many seconds the container execution shall be
terminated. The default value is specified by the constant ``TIMEOUT'' as a fallback since we do not want any
container action to last forever.\\
Also not essential but highly recommended is the option ``use\_template\_container'' (default: None). This
requires an already existing template container from which the daemon clones a new temporary one. Not only
this is much faster than creating a container from scratch every time, but it is also an easy way to tweak
the containers for ones purposes, for example through installing additional software packages.\\
The returned json dictionary always contains a boolean variable ``success'' which if set to ``True'' should
furthermore contain any results, or if set to ``False'' some error information under the keyword ``info''.\\
The ``run\_prog'' action returns the exit value of the program and a boolean variable ``timeout'' to indicate
whether the container terminated normally or had to be shut down by the daemon.

\section{The config template}

During container
creation a generic config file is created which afterwards will be overwritten by the lxc\_daemon using
the template.\\
The daemon is aware of two keywords within the config template:\\
The keyword ``\{container\}'' is replaced with the actual file path of the container. Note that every
subdirectory (e.g. the rootfs) needs to be preceded by a slash.\\
The keyword ``\{name\}'' is simply replaced with the container name.\\
The contents of this config are explained in chapter ``Security''.

\section{Container handling}

In the following a top--down examination regarding the container handling is described:\\
When the server socket detects an incoming connection and creates a thread in which ``handleClient'' is responsible
for the communication to the corresponding program, a container object is already created. This is not accompanied by
any other action, the container does not exist yet. It is simply done to make the function and any other functions
called in the further process - if desired - aware of the container.
In addition to that this function is the only one in this particular thread which persists as long as the
client is connected to the daemon and is therefore best suited for holding a reference to a container.\\
The function ``actionSupervisor'' takes the container object from ``handleClient'' and returns a response dictionary.
The purpose of this function is to make sure the container exists and if not is created accordingly and destroyed
properly when the connection shall no longer be kept alive. It also calls the function which determines the action
to be performed and thus which function is to be called respectively.\\
Since the only defined action is ``run\_prog'' at the moment it is sufficient to describe only this way although it is
non-deterministic in principle.\\
The ``handleRunProgAction'' function decodes the base64 string into ordinary byte code and passes it along with the
request dictionary and the container object to the function ``processLXC'' which then returns a tuple containing the return
value of the executed program, a boolean which indicates whether a timeout occurred during the process and a string
with error information which is not assigned if no error occured. The further handling of the return code needs some
explanation:\\
% source dammit -.-
The exit code of a program is an unsigned 8 bit integer. The attach\_wait method returns a 16 bit integer though. The 8
most significant bits are the exit code of the executed program while the 8 least significant bits indicate the exit
value of the attach\_wait process. If the latter fails it returns -1 which will result in the actual return value of
255 due to its unsigned nature. As a consequence any value returned by the program which is something other than 0 will be
above 255 which is why the daemon does not return the exit value directly but rather computes it as exit code modulo
255.\\
Afterwards a dictionary is created and the information is put into it. If ``info'' is not none ``success'' ist set to false
and the info string is appended to the dictionary.\\
The ``processLXC'' function starts the container, starts a timer, writes the binary code into the root filesystem, runs
``attach\_wait'', then tries to stop the timer, stops the container and in the end returns the tuple of arguemnts
described above.\\
Everything from the container start to the attach\_wait method is in a try-block to be be able to provide information on
any unexpected error and most importantly make sure that the container is stopped. Not having the container in a stopped
state will break the whole process in such a way that it will leave us with an orphaned container since running containers
cannot be destroyed.\\
Directly after the start a timer thread is started which will notify the current thread through a timer event if the
timeout is triggered and the container is stopped externally by the timer thread. This makes sure that the container
execution thread will eventually terminate.\\
After that the binary file is written to the rootfs and made executable (file modes and other metadata are not preserved).\\
The attach\_wait method executes the program inside the container and waits for the exit status\cite{attachwait}. The first argument is
one of two function handles (the other one being ``lxc.attach\_run\_shell''). Either one of those needs to be specified. Then
follow the parameters to pass to this function in a list object and finally a keyword argument is given. The
``attach\_flags'' are a combination of bitwise or'ed flags which determine the behaviour of the attach process. Among
the default options are capability dropping and moving the process to cgroups, furthermore setting the personality
(main purpose in lxc: running 32 bit containers on 64 bit hosts)\cite{personality} and enabling linux security modules
which has no effect with the current configuration neither using App Armor nor any other lsm.\\
In the ``finally'' block the container is stopped except if the timeout event is set. In that case the code makes use of
the container object's own function to wait for a certain state. This is done to avoid possible race conditions since
the container is stopped by another thread and we definitely do not want to try to destroy the container before it
reaches the stopped state. The returning of the processLXC function guarantees to have a stopped container.