\section{Motivation}

\begin{frame}
  \frametitle{Motivation}
  \begin{itemize}
  \item Ziel: Schutz vor nicht vertrauenswürdigen Programmen
  \begin{itemize}
   \item Testumgebung
   \item unbekannte Programme
   \item Sicherheit für Server
  \end{itemize}
  \item sicheres, automatisches Testen der C-Projekte in Informatik 2
  \end{itemize}
\end{frame}


\begin{frame}
  \frametitle{Lösungen}
  \begin{itemize}
    \item{Virtual Machine}
    \begin{itemize}
      \item{Emulation von Hardware}
      \item{separates Betriebssystem}
    \end{itemize}
    \item{Ptrace}
    \begin{itemize}
      \item{überwacht System Calls}
    \end{itemize}
    \item{Container Virtualization}
    \begin{itemize}
      \item separates Datenset
      \item nutzt den Host-Kernel
      \item isoliert durch Kernel-Funktionen
    \end{itemize}
  \end{itemize}
\end{frame}


\section{Grundlagen}

\begin{frame}
  \frametitle{Linux Containers (Lxc)}
  \begin{itemize}
    \item leichtgewichtig
    \begin{itemize}
      \item Erstellung etwa 10s
      \item Start/Stop etwa 1s
      \item Speicherverbrauch im Leerlauf etwa 3MB
    \end{itemize}
    \pause
    \item existieren als \textit{rootfs} auf dem Host System
    \begin{itemize}
      \item rootfs enthält sämtliche Dateien zur Ausführung eines Betriebssystems
      \item Ausnahme in Lxc: Kernel
    \end{itemize}
    \pause
    \item isoliert auszuführende Programme vom Rest des Systems durch:
    \begin{itemize}
      \item Chroot
      \item Kernel Namespaces (PID, mount...)
      \item Cgroups (memory, cpu...)
      \item Linux Capabilities (cap\_sys\_boot, cap\_sys\_chroot)
      \item Linux Security Modules
      \item Seccomp Policies
    \end{itemize}
  \end{itemize}
\end{frame}